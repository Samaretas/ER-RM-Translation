\documentclass[class=book, crop=false, oneside]{standalone}
\usepackage[subpreambles=true]{standalone}

\usepackage{import}

\graphicspath{{./assets/diagrams/}}

\begin{document}

\chapter*{Translating ER into RM}
The best way to learn this mechanism is to poceed by examples.\\
Let's start with this one:

\begin{figure}[H]
	\centering
	\includegraphics[width=.9\textwidth,keepaspectratio]{diagram1.png}
	\caption{}
	\label{diagram1}
\end{figure}

When we want to translate an ER diagram into an RM model we have to follow certain rules, let's check them out.
\paragraph*{Rule \#1} every entity becomes a table and its attributes are the table's columns.
\paragraph*{Rule \#2} each relation becomes a table and its attributes are part of the table's columns, also the keys of the entities involved in the relation become columns and they are flagged as Foreign Keys(FK).

\vskip 20pt

\begin{minipage}{0.45\textwidth}

	\begin{table}[H]
		\centering
		\subimport{assets/tables/}{phc-pers.tex}
	\end{table}

\end{minipage}
\hspace{.1\textwidth}
\begin{minipage}{.45\textwidth}

	\begin{table}[H]
		\centering
		\subimport{assets/tables/}{phc-cars}
	\end{table}

\end{minipage}

\vskip 20pt
\begin{minipage}{\textwidth}

	\begin{table}[H]
		\centering
		\subimport{assets/tables/}{phc-has}
	\end{table}

	\end{minipage}
	
	\vskip 20pt

	\begin{minipage}{\textwidth}

		\begin{table}[H]
			\centering
			\subimport{assets/tables/}{phc-cars_v2}
		\end{table}

	\end{minipage}



\end{document}
